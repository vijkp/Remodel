\documentclass[12pt,letterpaper,titlepage, one-sided]{article}
\usepackage[top=3cm, bottom=3cm, left=3cm, right=3cm]{geometry}
\usepackage{changepage}


% Title Page
\title{Remodel\\
(Systems Course Project)\\
Fall 2013}
\author{Submitted by\\
Vijay Kumar Pasikanti\\
\texttt{(vijaykp@cs.umass.edu)}\\
\\
\\
School of Computer Science\\
University of Massachusetts Amherst
}

\begin{document}
\newcommand{\remodel}{\texttt{remodel }}
\newcommand{\shellcmd}[1]{\\\indent\indent\texttt{#1}}
\maketitle
\tableofcontents
\newpage
\section{Introduction}
  \texttt{Remodel} is a replacement program for GNU \texttt{make}. \texttt{make} takes a series of dependencies and builds targets by executing commands. Its use of UNIX timestamps as a happens-before relation means that it can fail to build properly. This is especially true during parallel builds. \\ \\
  \texttt{Remodel} also takes a series of dependencies and targets, etc. (grammar below), but unlike make, it will use MD5 hashes to detect new content and provide, together with the dependency chain, a happens-before relation (md5diff(oldA,newA) => everything that depends on A must be rebuilt). It executes all independent (non-conflicting) commands in parallel, using multiple threads. 
  
	\subsection{Grammar}
	\remodel uses a different grammar than \texttt{make}. Dependencies can appear in any order. If you execute remodel with no arguments, it should start with the pseudo-target DEFAULT. Otherwise, the root is the argument to remodel, as in remodel foo.o. 
	\begin{adjustwidth}{.6in} {0pt}
	\texttt{\\
    program \ \ \ ::= production*\\
    production ::= target '<-' dependency (':' '"' command '"")\\
    dependency ::= filename (',' filename)*\\
    target\ \ \ \ \ ::= filename (',' filename)*\\
    }
	\end{adjustwidth}
	Here's an example that builds the program baz from two source files, foo.cpp and bar.cpp. 
	\begin{adjustwidth}{.6in} {0pt}
	\texttt{\\
    DEFAULT <- baz\\
    baz\ \ \ \ \ <- foo.o, bar.o: "g++ foo.o bar.o -o baz"\\
    foo.o  \ \ <- foo.cpp : "g++ -c foo.cpp -o foo.o"\\
    bar.o \ \ <- bar.cpp: "g++ -c bar.cpp -o bar.o"\\
 	}
	\end{adjustwidth}
	\subsection{Platform and dependencies}
	\begin{itemize}
		\item Platform: linux  
		\item Dependencies: \texttt{openssl, glibc} (package name \texttt{libssl-dev})
		\item Compiler: \texttt{gcc 4.6}
		\item Programming Language: \texttt{C}
	\end{itemize}

\section{Implementation}
\remodel is completely written in 'C' language using standard \texttt{glibc} and \texttt{openssl} libraries and works for any flavor of Linux OS. \remodel uses multiple threads defined by the user while building files which are non-dependent on each other.
\\ \\
\remodel computes MD5 hash string of each source file to identify the changed files. After the build MD5 hash strings are stored in a file \texttt{md5hases} within the directory \texttt{.remodel} at the root directory level of the project that was built. File \texttt{.remodel/md5hashes} is read and all the hash strings are loaded into \remodel at the beginning of its execution.

	\subsection{Algorithm}
		\begin{enumerate}
		\item
		Validate arguments.
		\item
		Read and parse \texttt{Remodelfile}. Make lists of targets and source files.
		\item
		Read \texttt{.remodel/md5hashes} file and load hash strings from the previous build.
		\item
		Generate dependency tree starting from the given \texttt{target}.
		\item
		Generate MD5 hash strings of each source file. Compare and identify the files that changed. If the build is successful, the new MD5 hash strings are saved to disk.
		\item
		Mark changed source files and targets that need build in the \textit{dependency tree} and generate a reduced tree by removing the unnecessary targets.
		\item
		Start $N$ \textit{build threads} for parallel builds and a \textit{monitor thread}. $N$ is calculated based on the number of targets that need build using $N = \sqrt{targets}$
		\item
		Starting from the leafs of the \textit{dependency tree} place each target in a \textit{dispatch queue} where they are picked up by \textit{build threads}. \textit{Build threads} send an acknowledgment to \texttt{main} program as soon as the build is done for the target it processed.
		\item
		Based on the responses from the \textit{build threads} and \textit{monitor thread} \texttt{main} program processes the \textit{dependency tree} to choose the next target to build and places it in the \textit{dispatch queue}.
		\end{enumerate}
		\subsection{Meta files}
		\begin{itemize}
		\item
		\remodel uses a meta file \texttt{md5hashes} to store md5 hash strings of source files. It resides inside the directory \texttt{.remodel} in project's root directory.
		\item
		The format it uses is \texttt{<src file name> <md5 hash string>}. For example: 
		\shellcmd{sched.c 105fcbfaf254826be856c5f166f76e21}
		\shellcmd{network.h 379058d4f4738993f6e37a60f448e36a}
		\shellcmd{network.c 8586deee068420812704a88a7274f91b}
		\shellcmd{filesystem.h 20e986d00cadeb329b0143cf95319548v}
		
		\item
		MD5 hashes are read in the beginning of \remodel execution to compare with the current source file MD5 hashes.
		\item
		MD5 hashes calculated in the current build are stored back into the meta file only if the build is successful.
		
		\end{itemize}
		\subsection{Threads for parallel builds}
		Threads are created and run using pthread library. There are two types of threads used in this program viz. 1.\textit{build thread} and 2.\textit{monitor thread}. Each type of thread is described below. 
		
		\subsubsection{Build Thread} 
		\begin{itemize}
		\item
		\textit{Build threads} are created before the start of the building process. Number of \textit{build threads} are \textit{N} and \textit{N} is chosen based on the number of targets that needs build. If \textit{targets} are the number of targets that needs build then \textit{N} is given by the equation $N = \sqrt{targets}$.
		\item
		\textit{Build threads} would be running until the whole build process is finished. \texttt{main} program will send a kill message to \textit{build threads} to terminate when the build is done.
		\item
		As soon as a \textit{build thread} is done with the build command execution for a target, it sends a completion message to \texttt{main} using \textit{response queue}.
		\item
		{Build threads} poll on \textit{dispatch queue} looking for targets to build. \texttt{main} program dispatches targets to build threads via \textit{dispatch queue}. \textit{Dispatch queue} is explained in detail in the queues section.
		\end{itemize}
		
		\subsubsection{Monitor Thread}
		\begin{itemize}
		\item
		\textit{Monitor thread} is a single thread. It is created when build threads are created and terminated when the build process is finished. The main function of \textit{monitor thread} is to monitor the targets that appear as a dependency for multiple targets.
		\item
		As soon as a target is built by a \textit{build thread} the corresponding	target that is being monitored is sent to \texttt{main} via \textit{response queue}
		\end{itemize}
	
		\subsubsection{\texttt{main}}
		\begin{itemize}
		\item
		\texttt{main} handles responses from \textit{build threads} and \textit{monitor thread} via \textit{response queue}. For each message in \textit{response queue} \texttt{main } updates appropriate build completion flags for each target and dependency. As soon as \texttt{main} figures out that all dependencies for a target are built, the target is scheduled for build by being placed in \textit{dispatch queue}.  
		\end{itemize}
		
		\subsection{Queues for inter-thread communication}
		\remodel uses queues for message passing across multiple threads to communicate among each other. For \remodel queues are implemented from scratch using double-link lists. \remodel uses 3 queues viz. 1. \textit{dispatch queue} 2.\textit{monitor queue} 3.\textit{response queue}. Each queue is described in detail below.
		\subsubsection{Dispatch Queue}
		\begin{itemize}
		\item
		\textit{Dispatch queue} holds the targets that are ready for build. \textit{Build threads} pick up a targets from \textit{dispatch queue }and execute the build command.
		
		\item
		\texttt{main} fills up the queue, \textit{build threads} and \textit{monitor thread} drain the queue.
		\end{itemize}
		
		\subsubsection{Monitor Queue}
		\begin{itemize}
		\item
		\textit{Monitor queue} is used by \textit{monitor thread} as a place holder for monitoring targets that are dependencies for multiple targets. \textit{Monitor thread} is the only producer and consumer of the queue.
		\end{itemize}
		
		\subsubsection{Response Queue}
		\begin{itemize}
		\item
		\textit{Response queue} is a place holder for responses from \textit{build threads} and \textit{monitor queue}. \texttt{main} handles each response from the \textit{response queue}. \textit{build} and \textit{monitor} threads fill up the queue and \texttt{main} removes elements from the queue.
		\end{itemize}
		
\section{Usage}
	\subsection{Writing Remodelfile}
	\begin{enumerate}
	\item
	All target definitions should be place in a file with the name \texttt{Remodelfile} or \texttt{remodelfile} and it should be placed in the root directory of the project that is being built.
	\item
	Target definitions should written according the grammar given in the introduction section and they can appear in any order with in the file. 
	\item
	Each target definition should be written in a single line.
	\item
	One of the targets should be defined as \texttt{DEFAULT}. \remodel uses \texttt{DEFAULT} as the default target when none specified in the command.
	\item
	\remodel parses \texttt{Remodelfile} and checks for any cyclic dependencies before proceeding with the build.
	\end{enumerate}
	
	\subsection{CLI}
	\remodel shoudl be run from within the project directory which has the \texttt{remodelfile} or \texttt{Remodelfile}. 
	\remodel command takes \texttt{target name} as an argument. If no arguments are specified it used \texttt{DEFAUTLT} as the argument.
	\\
	To show help of the command use \texttt{-h} or \texttt{--help}. For example:\\
	\shellcmd{[vijaykp@linux test\_project]\$ ../remodel --help}
	\shellcmd{remodel: Usage: remodel [<target\_name> | --help | -h]}
	\shellcmd{\ \ \ \ \ \ \ \ \ \ Description: <target\_name> is optional.} 
	\shellcmd{\ \ \ \ \ \ \ \ \ \ \ \ \ \ \ \ \ \ \ \ \ \ \ \ Default value 'DEFAULT'}
	\shellcmd{\ \ \ \ \ \ \ \ \ \ \ \ \ \ \ \ \ \ \ \ \ \ \ \ is used when none specified.}                     
	\shellcmd{\ \ \ \ \ \ \ \ \ \ Options: '-h' and '--help' will print this message}
\section{Build and test \remodel}
	\begin{enumerate}
	\item
	Change directory to \remodel directory.
	\item
	Before building \remodel make sure that \texttt{libssl-dev} and \texttt{gcc} are installed.
	\item
	Run \texttt{make} for building \remodel for testing. \\Alternatively, use \texttt{make debug} command to build a debug build. Debug build is useful while debugging. It builds the project with gdb flags and enables debug log messages. Make sure to run \texttt{make clean} before switching from \texttt{debug} to \texttt{prod} build and vice versa.
	\item
	Build generates the binary \remodel. 
	\item
	Change to any test directory and run \remodel either by copying \remodel binary to test directory or providing the full path to \remodel.
	\end{enumerate}
\subsection{Testcases}
	Test cases are provided in \texttt{testcases/} directory. Shell scripts in \texttt{testscripts/} can be used for testing test cases in \texttt{testcases/} directory.
\end{document}          
